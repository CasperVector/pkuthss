% vim:ts=4:sw=4
%
% Copyright (c) 2008-2009 solvethis
% Copyright (c) 2010-2015 Casper Ti. Vector
% Public domain.
%
% 使用前请先仔细阅读 pkuthss 和 biblatex-caspervector 的文档,
% 特别是其中的 FAQ 部分和用红色强调的部分。
% 两者可在终端/命令提示符中用
%   texdoc pkuthss
%   texdoc biblatex-caspervector
% 调出。

% 采用了自定义的(包括大小写不同于原文件的)字体文件名,
% 并改动 ctex.cfg 等配置文件的用户请自行加入 nofonts 选项;
% 其它用户不用加入 nofonts 选项,加入之后反而会产生错误。
%
% 图书馆要求电子版论文的目录必须为黑色,
% 且某些教务要求打印版论文的文字部分为纯黑色而非灰度打印,
% 【因此最终打印和提交论文前,请将“colorlinks”改为“nocolorlinks”。】
\documentclass[UTF8, colorlinks]{pkuthss}

% 使用 biblatex 排版参考文献,并规定其格式(详见 biblatex-caspervector 的文档)。
% 这里按照英文文献在前,中文文献在后排序(“sorting = ecnty”);
% 若需按照中文文献在前,英文文献在后排序,请设置“sorting = centy”;
% 若需按照引用顺序排序,请设置“sorting = none”。
\usepackage[backend = biber, style = caspervector, utf8, sorting = none]{biblatex}

% 按学校要求设定参考文献列表中的条目之内及之间的距离。
\setlength{\bibitemsep}{3bp}
% 对于 linespread 值的计算过程有兴趣的同学可以参考 pkuthss-extra.sty。
\renewcommand*{\bibfont}{\zihao{5}\linespread{1.27}\selectfont}

% 设定文档的基本信息。
\pkuthssinfo{
	cthesisname = {博士研究生学位论文}, ethesisname = {Doctor Thesis},
	ctitle = {测试文档}, etitle = {Test Document},
	cauthor = {某某},
	eauthor = {Test},
	studentid = {0123456789},
	date = {某年某月},
	school = {某某学院},
	cmajor = {某某专业}, emajor = {Some Major},
	direction = {某某方向},
	cmentor = {某某教授}, ementor = {Prof.\ Somebody},
	ckeywords = {其一,其二}, ekeywords = {First, Second}
}
% 载入参考文献数据库(注意不要省略“.bib”)。
\addbibresource{thesis.bib}

% 普通用户可删除此段。
\usepackage{color}
\def\pkuthssffaq{%
	\emph{\textcolor{red}{pkuthss 文档模版最常见问题:}}

	在最终打印和提交论文之前,
	请将 pkuthss 文档类选项中的 %
	\texttt{colorlinks} 替换为 \texttt{nocolorlinks},
	因为图书馆要求电子版论文的目录必须为黑色,
	且某些教务要求打印版论文的文字部分为纯黑色而非灰度打印。

	\texttt{\string\cite}、\texttt{\string\parencite} %
	和 \texttt{\string\supercite} 三个命令分别产生%
	未格式化的、带方括号的和上标且带方括号的引用标记:%
	\cite{test-en},\parencite{test-zh}、\supercite{test-en, test-zh}。

	若要避免章末空白页,请在调用 pkuthss 文档类时加入 \texttt{openany} 选项。

	如果编译时不出参考文献,
	请参考 \texttt{texdoc pkuthss}“问题及其解决”一章
	“其它可能存在的问题”一节中关于 biber 的说明。
}

\begin{document}
	% 以下为正文之前的部分,默认不进行章节编号。
	\frontmatter
	% 此后到下一 \pagestyle 命令之前不排版页眉或页脚。
	\pagestyle{empty}
	% 自动生成封面。
	\maketitle

	% 此后到下一 \pagestyle 命令之前正常排版页眉和页脚。
	% 封面要求单面打印,故需新开右页,此处已一并实现。
	\cleardoublepage
	\pagestyle{plain}
	% 重置页码计数器,用大写罗马数字排版此部分页码。
	\setcounter{page}{0}
	\pagenumbering{Roman}

	% 版权声明。
	\include{chap/copyright}
	% 中英文摘要。
	\include{chap/abstract}
	% 自动生成目录。
	\tableofcontents

	% 以下为正文部分,默认要进行章节编号。
	\mainmatter
	% 序言。
	\include{chap/introduction}
	% 各章节。
	% Copyright (c) 2014,2016,2018 Casper Ti. Vector
% Public domain.

\chapter{引言}
% \pkuthssffaq % 中文测试文字。

% vim:ts=4:sw=4

	% 结论。
	\include{chap/conclusion}

	% 正文中的附录部分。
	\appendix
	% 排版参考文献列表。
	\printbibliography[
		% 使“参考文献”出现在目录中;如果同时要使参考文献列表参与章节编号,
		% 可将“bibintoc”改为“bibnumbered”。
		heading = bibintoc,
		% 单独设定排序方案。此设定会局部覆盖之前的全局设置。
		% 注:只有同时使用 2.x 或之后版本的 biblatex 和相应兼容版本的 biber,
		% 才能对每个 \printbibliography 命令采用不同的排序方案,
		% 否则只能在载入 biblatex 宏包时就(全局)指定排序方案。
		% 在这样的情况下,请去掉所有的 sorting 选项,否则可能出错。
		% 此外,biblatex 3.0 中 \printbibliography 的 sorting 选项失效,
		% 详见 biblatex-caspervector 的文档。
		sorting = ecnty
	]
	% 各附录。
	\include{chap/encl1}

	% 以下为正文之后的部分,默认不进行章节编号。
	\backmatter
	% 致谢。
	\include{chap/acknowledge}
	% 原创性声明和使用授权说明。
	\include{chap/originauth}
\end{document}

