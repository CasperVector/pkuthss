\documentclass[UTF8]{pkuthss}

% 提供 Verbatiminput 命令。
\usepackage{fancyvrb}

% 此命令专用于排版包含在 pkuthss 文档类源文件中的代码。其中前 26 行为共用的
% 头部(主要为版权声明),故没有列出。
\newcommand{\VerbatimCode}{\VerbatimInput[%
	frame=lines,fontsize=\small,tabsize=2,
	baselinestretch=1.25,firstline=27,numbers=left
]}

% 参考文献格式。
\bibliographystyle{ref/chinesebst}
% 文档类版本。
\newcommand{\docversion}{v1.3 alpha1}
% 设定文档的基本信息。
\pkuthssinfo{
	cthesisname={本科生毕业论文},ethesisname={Undergraduate Thesis},
	ctitle={北京大学论文文档模板\\pkuthss \docversion},
	etitle={The PKU dissertation document class\\pkuthss \docversion},
	cauthor={盖茨波 $\cdot$ 钛 $\cdot$ 维克托},
	eauthor={Casper Ti.\ Vector},
	studentid={OXXXXXXX},
	date={二〇一一年一月},
	school={化学与分子工程学院},
	cmajor={化学},emajor={Chemistry},direction={理论和计算化学},
	cmentor={XX 教授},ementor={Prof.\ XX},
	ckeywords={\LaTeX2e,排版,文档类,\CTeX},
	ekeywords={\LaTeX2e, typesetting, document class, \CTeX},
}
\begin{document}
	%% 以下为正文之前的部分。
	\frontmatter
	% 自动生成标题页。
	\maketitle
	% 版权声明。
	\cleardoublepage
\chapter*{\textbf{版权声明}}
{
	\zihao{3}
\begin{comment}
	任何收存和保管本论文各种版本的单位和个人,
	未经本论文作者同意,不得将本论文转借他人,
	亦不得随意复制、抄录、拍照或以任何方式传播。
	否则一旦引起有碍作者著作权之问题,将可能承担法律责任。
\end{comment}
	版权所有~\copyright~1991-2010 Casper Ti. Vector

	本文档可在~GNU~自由文档许可证(GFDL)\footnote%
	{\ \url{http://www.fsf.org/licensing/licenses/fdl.html}}%
	的第~1.3~版(或之后任意版本)或~GNU~通用公共许可证(GPL)\footnote%
	{\ \url{http://www.fsf.org/licensing/licenses/gpl.html}}%
	的第~3~版(或之后任意版本)所规定的条款下自由地复制、修改和发布。
	
	以上所述两个许可证应该在本文档所在目录的~\verb|license/|~%
	\linebreak[1]子目录下,
	文件名分别为~\verb|fdl-1.3.txt|~和~\verb|gpl-3.0.txt|。
	如果没有,你可以到上面提到的网址查看许可证内容。
	如果还不行,请写信给下面的地址以获得邮寄的许可证:
\begin{verbatim}
    The Free Software Foundation, Inc.,
    675 Mass Ave, Cambridge, MA02139, USA 
\end{verbatim}
	\par
}


	% 中英文摘要。
	% 摘要要求在 3000 字以内。

\cleardoublepage
\begin{cabstract}

	本文介绍了~\emph{pkuthss}~这个文档模板所提供的功能,
	并以自身为例演示了该模板的使用。

\end{cabstract}

\cleardoublepage
\begin{eabstract}

	This paper describes the the functions provided by
	the \emph{pkuthss} document template,
	and provides itself as an example to illustrate
	the usage of the document class.

\end{eabstract}


	% 自动生成目录。
	\tableofcontents

	%% 以下为正文。
	\mainmatter

	% 绪言。
	\specialchap{绪言}

本文档是“北京大学论文文档模板”的测试和说明文档。

以前的学位论文模板工作由包括~%
dypang\supercite{dypang}、FerretL\supercite{FerretL}、
lwolf\supercite{lwolf}、Langpku\supercite{Langpku}、
solvethis\supercite{solvethis}~%
的数人做过。
本论文模板是~solvethis~的~pkuthss~模板的更新版本,
更新的重点是重构和对新文档类、宏包的支持。

pkuthss~文档模板现在的维护者是~Casper Ti. Vector\footnote%
{\ \href{CasperVector@gmail.com}{\texttt{CasperVector@gmail.com}}}。


	% 各章节。
	\chapter{使用介绍}
	\section{重要文件}

	本文档所在目录下各重要文件如下:
	\begin{itemize}\denselist
		\item \verb|pkuthss.cls|:pkuthss~文档类的类文件。
		\item \verb|pkuthss.def|:在~\verb|pkuthss.cls|~中使用的定义文件。
		\item \verb|sample.tex|:主文件,编译该文件即可。
		\item \verb|sample.pdf|:即本文档,由编译~\verb|sample.tex|~得到。
		\item \verb|Makefile|:Makefile,用于使编译工作自动化。
		\item \verb|Make.bat|:
			Windows~下的伪“Makefile”,由~Windows~的批处理实现。
		\item \verb|chap/|:文件夹,包含各章节内容:
		\begin{itemize}\denselist
			\item \verb|copyright.tex|:版权声明部分\footnote%
			{%
				因为本文档的许可证限制,我们必须附上许可证的文本;
				但用户可能选择其它类型的版权声明,
				故~\texttt{license/}\linebreak[1]~目录不是必需的。
				一个可能更常用的版权声明已经放在此文件中,但被注释掉了,
				用户可以考虑使用那个版本。
				如果使用那个版本,就不再需要~\texttt{license/}~目录了。
			}。
			\item \verb|originauth.tex|:
				原创性声明和使用授权说明部分~\supercite{F11}。
		\end{itemize}
		\item \verb|img/|:文件夹,包含论文中所有图片:
		\begin{itemize}\denselist
			\item \verb|Makefile|:图片部分的~Makefile。
			\item \verb|Make.bat|:
				Windows~下的伪“Makefile”,由~Windows~的批处理实现。
			\item \verb|pkulogo.ps|:北大校徽。
			\item \verb|pkuword.ps|:“北京大学”字样。
		\end{itemize}
	\end{itemize}

	\section{系统要求}

	正确编译需要以下几部分:
	\begin{itemize}\denselist
		\item 一个基本的~\LaTeX{}~发行版。
		\item CJK~或~xeCJK(供~Xe\LaTeX{}~使用)宏包。
		\item ctex~宏包\supercite{ctex-doc,ctexfaq}%
			(提供了~ctexbook~文档类)。
		\item 中文字体。
		\item 如果需要使用~Makefile~来实现自动编译,还需要~Make~工具;
			但如果使用由批处理实现的伪“Makefile”就不用了。
	\end{itemize}

	最新的~\TeX{}Live~系统和~\CTeX~套装都已经包含%
	除中文字体之外所有要求的项目;中文字体需要用户自行获得。

	Linux~用户可以从软件源获得~GNU~的~make;
	其它类~UNIX~系统应该也会提供~make~工具,请参阅相应的文档以获得帮助。
	Windows~用户可以从以下地址下载~Windows~下的~GNU make~工具:

	\url{http://gnuwin32.sourceforge.net/packages/make.htm}(国际网)
	\vspace{-0.1em}\par
	\url{http://c.pku.edu.cn/software/c/mingw-c.7z}\footnote%
	{\ 感谢曹东刚老师在教学网站提供~GNU make~的下载。}(北大校园网)

	为了获得最好的支持,我们建议用户使用最新版的~\LaTeX{}~系统和各宏包。

	\section{编译方式}

	pkuthss~文档模板支持三种编译方式,即
	\begin{itemize}\denselist
	  \item \LaTeX{} -- dvipdf~方式:
		即顺次执行~\verb|latex|,\verb|bibtex|,%
		\verb|latex|,\verb|latex|,\verb|dvipdfmx|。
	  \item pdf\LaTeX{}~方式:
		即顺次执行~\verb|pdflatex|,\verb|bibtex|,%
		\verb|pdflatex|,\verb|pdflatex|。
	  \item Xe\LaTeX{}~方式:
		即顺次执行~\verb|xelatex|,\verb|bibtex|,%
		\verb|xelatex|,\verb|xelatex|。
	\end{itemize}

	pkuthss~文档模板附带的~Makefile~中已经对这三种编译方式进行了完整的配置。
	用户只需要在~Makefile~中通过设定变量~\verb|JOBNAME|~的值%
	指定被编译的主文件名,
	并通过设定变量~\verb|LATEX|~的值指定采用哪种编译方式,
	即可通过在主文件所在目录调用~Make~工具来实现自动编译:
	如果是在类~UNIX~环境下,则用户应该调用的命令名为~\verb|make|;
	而如果是在~Windows~环境下,
	则用户应该调用的命令名为~\verb|mingw32-make|。

	用户如果不想配置~Windows~下的~GNU Make,
	则也可以使用由~Windows~批处理实现的伪“Makefile”,
	通过在主文件所在目录调用~\verb|make|\footnote%
	{\ %
		Windows~将批处理文件作为可执行文件,
		调用时可以不显式地指出扩展名。
	}~或直接双击~\verb|make.bat|~的图标运行之。%
	\emph
	{%
		注意:这样不能自动生成编译所需的部分图片。
		用户可能需要进入~\texttt{img/}~目录%
		执行那里的~\texttt{make.bat}~来手动生成这些图片。
	}


	\chapter{pkuthss~文档类提供的功能}
	\section{文档类选项}

	pkuthss~文档类以~ctexbook~文档类为基础,
	其接受的所有文档类选项均被传递给~ctexbook。

	例如,ctexbook~文档类默认使用~GBK~编码,
	因此如果需要使用~UTF-8~编码撰写论文,
	则在导入~pkuthss~文档类时加上~\verb|UTF8|~选项即可:
\begin{verbatim}
      \usepackage[UTF8,...]{pkuthss} % “...”代表其它的选项。
\end{verbatim}
	类似地,如果需要使用~hyperref~宏包,
	则为了利用~ctexbook~文档类对~hyperref~宏包的支持特性,
	可以传递~\verb|hyperref|~选项给~ctexbook~宏包:
\begin{verbatim}
      \usepackage[hyperref,...]{pkuthss} % “...”代表其它的选项。
\end{verbatim}

	\section{pkuthss~文档类定义的命令}
		\subsection{用于设定文档信息的命令}

		这一类命令的语法是
\begin{verbatim}
      \commandname{具体信息} % commandname 为具体命令的名称。
\end{verbatim}

		这些命令总结如下\footnote%
		{\ \ttfamily\songti%
			\string\title、\string\author~和~\string\date~%
			实际上是从~ctexbook~文档类继承来的。%
		}:
		\begin{itemize}\denseenum
			\item \verb|\title|:设定论文标题;
			\item \verb|\etitle|:设定论文英文标题;
			\item \verb|\author|:设定作者;
			\item \verb|\eauthor|:设定作者的英文名;
			\item \verb|\date|:设定日期;
			\item \verb|\studentid|:设定学号;
			\item \verb|\school|:设定学院;
			\item \verb|\major|:设定专业;
			\item \verb|\emajor|:设定专业的英文名;
			\item \verb|\direction|:设定研究方向;
			\item \verb|\mentor|:设定导师;
			\item \verb|\ementor|:设定导师的英文名;
			\item \verb|\keywords|:设定关键词;
			\item \verb|\ekeywords|:英文关键词。
		\end{itemize}

		例如,如果要设定专业为“化学”,则可以使用以下命令:
\begin{verbatim}
      \major{化学}
\end{verbatim}

		\subsection{“\texttt{name}”类命令}

		这一类命令的语法是
\begin{verbatim}
      % commandname 为具体的命令名。
      \renewcommand{\commandname}{具体信息}
\end{verbatim}

		这些命令总结如下\footnote%
		{\ \ttfamily\songti%
			\string\contentsname~和~\string\bibname~%
			实际上是从~ctexbook~文档类继承来的。%
		}:
		\begin{itemize}\denseenum
			\item \verb|\thesisname|:论文类别名。
			\item \verb|\cabstractname|:中文摘要的标题。
			\item \verb|\eabstractname|:英文摘要的标题。
			\item \verb|\contentsname|:目录的标题。
			\item \verb|\bibname|:参考文献目录的标题。
		\end{itemize}

		例如,如果要设定论文的类别为“本科生毕业论文”,
		则可以使用以下命令:
\begin{verbatim}
      \renewcommand{\thesisname}{本科生毕业论文}
\end{verbatim}
		而如果要设定中文摘要的标题为“摘\hspace{2em}要”,
		则可以使用以下命令:
\begin{verbatim}
      \renewcommand{\abstractname}{摘\hspace{2em}要}
\end{verbatim}

		\subsection{其它命令}

		\begin{itemize}\denseenum
			\item \verb|\maketitle|:
				此命令根据设定好的文档信息自动生成论文的标题页,亦即封面。
			\item \verb|\specialchap|:
				此命令用于开始不进行标号但计入目录的一章,
				并合理安排其页眉。%
				\emph
				{%
					注意在此章内的节或小节等命令应使用带星号的版本,
					例如~\texttt{\string\section\string*}~等,
					以免造成章节编号混乱。%
				}
				\par
				例如,本文档中的“绪言”一章就是用~\verb|\specialchap{绪言}|~%
				这条命令开始的。
		\end{itemize}

		\subsection{从其它文档类和宏包继承的命令\label{sec:inherit}}

		pkuthss~文档类以~ctexbook~文档类为基础,
		并默认调用了以下宏包:
		\begin{itemize}\denseenum
			\item fntef:
				提供了~\verb|\maketitle|~中调用的~%
				\verb|\CJKunderline|~命令。
			\item graphicx\supercite{graphicx-doc}:提供图形支持。
			\item geometry\supercite{geometry-doc}:用于设置页面布局。
			\item fancyhdr\supercite{fancyhdr-doc}:用于设置页眉、页脚。
		\end{itemize}
		因此,ctexbook~文档类和这些宏包所提供的命令均可以使用。

		\vspace{1em}\par
		\emph
		{%
			注意:
			pkuthss~文档类中有一些一旦改动就有可能破坏预设的排版规划,
			因此不建议更改这些设置,它们是:
			\begin{itemize}\denseenum
				\item 纸张类型:A4;
				\item 版心尺寸:%
					$240\,\mathrm{mm}\times150\,\mathrm{mm}$,
					包含页眉、页脚;
				\item 默认字号:小四号。
			\end{itemize}
		}

	\section{pkuthss~文档类定义的环境}

	pkuthss~文档类定义了两个环境——\verb|cabstract|~和~\verb|cabstract|,
	分别用于编写中文和英文摘要。
	用户只需要写摘要的正文;标题、作者、导师、专业等部分会自动生成。

	此外,pkuthss~文档类还从第~\ref{sec:inherit}~节中所述的%
	文档类和宏包中继承了各种环境,用户也可以使用它们。


	\chapter{问题及其解决}
	\section{FAQ\label{sec:faq}}
	\begin{enumerate}
		\item[\textbf{Q:}]
		我的编译结果很奇怪,文字很靠近页面的顶端。请问这是怎么回事?

		\item[\textbf{A:}]
		请检查你的程序设置。
		如果使用~WinEdt,可点击~Options,选择~Execution Modes,
		检查一下~dvips、dvipdfmx、ps2pdf~等程序的纸张设置。

		\item[\textbf{Q:}]
		打印论文时不希望使用彩色的链接,请问应该怎么办?

		\item[\textbf{A:}]
		\verb|\hypersetup{colorlinks=false}|。
		关于书签和链接的问题,
		请参阅~hyperref~宏包的文档\supercite{hyperref-doc}。

		\item[\textbf{Q:}]
		导言区的内容好多,应该有好多在我的论文里是不必要的。
		请问可以去掉哪些?

		\item[\textbf{A:}]
		如果你使用~GBK~编码,则~pkuthss~文档类的~UTF-8~选项是不必要的。
		如果你不需要生成的~pdf~里的书签和链接,则~hyperref~宏包是不必要的,
		同时用于进行相关设置的~\verb|hypersetup|~命令也应该去掉。
		如果你不使用~\verb|\verbatiminput|~命令和~\verb|comment|~环境,
		则~verbatim~宏包是不需要的。
		如果你不需要上标的引用记号,则~\verb|\supercite|~宏可以去掉。
		如果你不需要使用密集的罗列环境,则~\verb|\denseenum|~宏可以去掉。

		wasysym~宏包不应该去掉,
		因为~\verb|chap/originauth.tex|~中使用了其提供的~\verb|\Box|~命令。
		设置页面居中和行距的命令不建议去掉:
		如果改变这些设置,虽然不会对排版效果造成致命的影响,
		但影响可能还是很显著的。

		\item[\textbf{Q:}]
		文档里面“致谢”一章的书签链接到的位置不对,请问这是为什么?

		\item[\textbf{A:}]
		这应该是由上游的~ctex~宏包的一个问题造成的。
		在~\verb|\backmatter|~以后用~\verb|\chapter|~命令开始的章节%
		也不会被编号,但会计入目录和产生书签。
		使用当前版本的~ctexbook~文档类的~pdf~文档%
		的这一类书签和链接指向的位置常常是错误的。
		这个问题应该正在修复中;在问题解决之前,
		一个缓解问题的办法是将~\verb|\backmatter|~以后%
		的~\verb|\chapter|~命令全部改为~\verb|\specialchap|~命令。
	\end{enumerate}

	\section{可能存在的问题}

	一个问题是~\ref{sec:faq}~中提到的书签的问题。
	这个问题应该很快能够得到解决。

	此外,还应该注意到,
	研究生手册\supercite{F13}和其电子版要求的论文封面并不一致。
	这里以电子版为准。

	\section{反馈意见和建议}

	关于~pkuthss~文档模板的意见和建议请到北大未名~BBS~的~MathTools~版提出,
	谢谢 :)


	% 结论。
	\begin{conclusion}

本文是北京大学博士(硕士)论文样式pkuthss.cls的测试和说明文档。
经过测试,该样式符合学校的有关要求,方便易用。这一工作对广大研
究生更好地撰写学位论文具有重要的意义。

\end{conclusion}


	\begin{appendix}
		% 参考文献。
		\bibliography{ref/pkuthss}
		% 各附录。
		\chapter{更新记录}

\verbatiminput{ChangeLog.txt}


		% Copyright (c) 2008-2009 solvethis
% Copyright (c) 2010-2011 Casper Ti. Vector
% Public domain.

\chapter{更新记录}

\raggedbottom % 避免某些奇怪的“Underfull \vbox”警告。
\VerbatimInput[tabsize=4]{ChangeLog.txt}
\flushbottom % 取消 \raggedbottom 的作用。


	\end{appendix}

	%% 以下为正文之后的部分。
	\backmatter

	% 致谢。
	\begin{acknowledgement}

感谢BDWM的Thesis版和MathTools版诸位同学的支持.

\end{acknowledgement}

	% 原创性声明和使用授权说明,不显示页码。
	\cleardoublepage\pagestyle{empty}
	\cleardoublepage
\section*{北京大学学位论文原创性声明和使用授权说明}
\vfill

\section*{原创性声明}

本人郑重声明:
所呈交的学位论文,是本人在导师的指导下,独立进行研究工作所取得的成果。
除文中已经注明引用的内容外,
本论文不含任何其他个人或集体已经发表或撰写过的作品或成果。
对本文的研究做出重要贡献的个人和集体,均已在文中以明确方式标明。
本声明的法律结果由本人承担。
\vspace{2.5em}\par

\rightline
{%
	论文作者签名:\hspace{5em}%
	日期:\hspace{2em}年\hspace{2em}月\hspace{2em}日%
}
\vfill

\section*{学位论文使用授权说明}
\vspace{-1em}\par
\centerline{\zihao{-4}(必须装订在提交学校图书馆的印刷本)}
\vspace{1em}\par

本人完全了解北京大学关于收集、保存、使用学位论文的规定,即:
\begin{itemize}\denseenum
	\item 按照学校要求提交学位论文的印刷本和电子版本;
	\item 学校有权保存学位论文的印刷本和电子版,
		并提供目录检索与阅览服务,在校园网上提供服务;
	\item 学校可以采用影印、缩印、数字化或其它复制手段保存论文;
	\item 因某种特殊原因需要延迟发布学位论文电子版,
		授权学校~\Square~一年~/\Square~两年~/\\
		\Square~三年以后在校园网上全文发布。
\end{itemize}
\par(保密论文在解密后遵守此规定)
\vspace{2.5em}\par

\rightline
{%
	论文作者签名:\hspace{5em}导师签名:\hspace{5em}%
	日期:\hspace{2em}年\hspace{2em}月\hspace{2em}日%
}


\end{document}

